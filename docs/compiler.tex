\documentclass[a4paper,10pt]{article}


%opening
\title{}
\author{}

\begin{document}

\maketitle

\begin{abstract}

Although the interesting and novel parts of the HotPy VM fcous on execution of bytecodes,
Python programs are not written in bytecode. The source code needs to translated to bytecode.
This translation is not complex, as should be easy to follow for anyone familiar with compilers.
Easily deduced details may be omitted.

\end{abstract}

\section{Translating Expressions}

\subsection{Variables}
Variables in Python are lexically scoped.
Local variables are assinged an index and translated as \verb|load_frame index|.
Global variables are translated as \verb|load_global idx|, where \verb|idx| is the index of the variable's name.

\subsection{Literals}
Most constants are translated as \verb|load_constant idx|, where\verb|idx| is the index of the value in the constant table.
Byte-sized integers are translated as \verb|byte val|. \verb|True|, \verb|False| and \verb|None| have their own bytecodes.

\subsection{Collections}

Collections of the form \verb|[a, b]| and \verb|{a:b, c:d}| are translated by first translating the members, then emitting a bytecode to construct the expression. Very large collections may be constructed in stages.

\subsubsection{Lists}
Lists of the form \verb|[a, b]|  are translated by first translating the members, then emitting the bytecode \verb|list n|.
Where \verb|n| is number of items in the list.

\subsubsection{Dicts}

Dicts are constructed by creating an empty \verb|dict| with the \verb|dictionary| bytecode. Values are added one key-value pair at a time.
For example \verb|{a:b, c:d}| translates as:
\begin{verbatim}
dictionary
code for a
code for b
dict_insert
code for c
code for d
dict_insert
\end{verbatim}

\subsection{Comprehensions}

\subsubsection{List Comprehensions}

List comprehensions are produced by first generating an empty list (with \verb|list 0|) and then iterating over the generating sequence:
the list comprehension \verb|[expr for var in seq]| translates as:
\begin{verbatim}
    code for seq
    iter
    store_frame temp_iter
    list 0
    store_frame temp_list
start-loop:
    list_for end
    load_frame temp_iter
    next
    store_frame var
    load_frame temp_list
    code for expr
    list_append
    end_loop start-loop
end:
    load_frame temp_list
\end{verbatim}

\subsubsection{Dictionary Comprehensions}

Dictionary comprehensions are translated in much the same way as list comprehensions. Replacing the \verb|list| with a \verb|dict| and substituting the \verb|dict_insert| bytecode for the \verb|list_append| bytecode.

\subsection{Attributes}

Attribute fetches are translated to the \verb|load_attr| bytecode. \verb|obj.attr| is translated as: \verb|(code for obj) load_attr idx|,
 where \verb|idx| is the index in the name table of \verb|attr|.

\subsection{Subscripting}
Subscripting is implemented with the  \verb|getitem| bytecode.
The expression \verb|a[b]| is translated as \verb|(code for a) (code for a) getitem|.

\subsection{Slicing}

Slicing is implemented as subscripting, with a slice object as the index.
\verb|a[:]| is translated as \verb|(code for a) None None None slice getitem|.
\verb|a[b:]| is translated as \verb|(code for a) (code for b) None None slice getitem|.
\verb|a[b:c:-1]| is translated as \verb|(code for a) (code for b) (code for c) byte -1 slice getitem|.

\subsection{Calls}

Calls are implemented with the \verb|call_simple n| and \verb|f_call| bytecodes.
Simple calls taking only positional parameters use \verb|call_simple|, other calls use \verb|f_call|.
\verb|f(a, b, c)| is translated as \verb|(code for f) (code for a) (code for b) (code for c) call_simple 3|.
For more complex constructions, the tuple and dictionary pass to \verb|f_call| are constructed explictly.
For example the call \verb|f(a, *t, b = 3)| is translated as:
\verb|(code for f) (code for a) (code for t) tuple_prepend empty_dict load_constant "b" byte 3 dict_insert f_call|

\subsection{Unary Operations}
Unary operations are implemented with the \verb|unary| bytecode. \verb|-x| is translated as \verb|(code for x) unary id|,
where \verb|id| is the ID of the negate operator. 

\subsection{Binary Operations}
Binary operations are implemented with the \verb|binary| bytecode. \verb|x + y| is translated as \verb|(code for x) (code for y) binary id|,
where \verb|id| is the ID of the addition operator. 

\subsection{Yield expressions}
The HotPy VM model requires that the stack depth in the generator function has a depth of exactly one when it yields.
The front end lifts yields to the top level. For example the expression \verb|x + (yield y) + z| is translated to:
\begin{verbatim}
_t0 = x
_t1 = yield y
(_t0 + _t1 + z)
\end{verbatim}
Once flattened the yield is translated to tehe \verb|yield| bytecode. \verb|yield y| translates to \verb|(code for y) yield|.

\subsection{Generator expressions}
Generator expressions are implemented as if the generator were implemented in a seperate function.
The generator expression \verb|(x+1 for x in seq)| is implemented as \verb|_gen(seq)|, where \verb|_gen| is implemented (in Python) as follows:
\begin{verbatim}
def _gen(seq):
    for x in seq:
        yield x+1
\end{verbatim}

\subsection{Evaluation Order}

Python has a script left to right evaluation order. This might require the front-end to insert a few stack manipulation bytecodes in order to get things to work properly.

\section{Translating Statements}

\subsection{Expression Statement}
An expression statement is simply any expression. It translated byt translating the expression then appending a \verb|drop| bytecode to discard the result.
\subsection{Assert Statement}
Assign statments are treated as no-ops if the global (immutable) value \verb|__debug__| is \verb|False|.
Otherwise \verb|assert cond| is equivalent to \verb|if not cond: raise AssertionError|.
\verb|assert cond, msg| is equivalent to \verb|if not cond: raise AssertionError(msg)|.
\subsection{Assignment Statement}
The right hand of an assignment is translated as an expression, above.
The right hand side is evaluated before the left hand side
The translation of the left hand depends on its form and generally follows the left hand side, but with \verb|load| replaced with \verb|store|.
\verb|var = None| is translated as \verb|None store_frame id| or \verb|None store_global id| depending on whether \verb|var| is local or global.
\verb|var.attr = None|  is translated as \verb|(code for var) None store_attr idx|.
\verb|var[0] = None|  is translated as \verb|None (code for var) 0 setitem|.
\verb|a, b = x| is translated as \verb|(code for x) unpack 2 store_frame id_a store_frame id_b|,
assuming \verb|a| and \verb|b| are both local variables.

\subsection{Augmented Assignment Statement}
Augmented assignements are statements of the form \verb|lhs += rhs|, which is equivlent to \verb|lhs = lhs + rhs|, except that augmented assignemnts use the \verb|inplace| bytecode rather than \verb|binary| and values should not be evaluated multiple times.
For example \verb|a[b] += c|, is the (roughly) the same as \verb|a[b] = a[b] + c|. Stack manipulations are used to avoid re-evalaution.
\verb|a[b] += c| translates as \verb|(code for c) (code for a) (code for b) two_copy getitem rotate4 inplace 'add' rrot setitem|
\subsection{Pass Statement}
Generate no bytecodes.
\subsection{Del Statement}

\subsection{Return Statement}
\verb|return expr| is translated as \verb|(code for expr) return|. A plain \verb|return| is equivelent to \verb|return None|.

\subsection{Raise Statement}
\verb|raise expr| is translated as \verb|(code for expr) raise|. A plain \verb|raise| statement raises the current exception, if any:
\verb|current_exception raise|.

\subsection{Break and Continue Statements}
The break and continue statements are implemented with the \verb|jump| bytecode; jumping to the end and start of the loop respectively.

\subsection{Import Statement}

\subsection{Global Statement}
Declares variable to be global, no bytecode is generated.
\subsection{Nonlocal Statement}
Declares variable to be non-local, no bytecode is generated.
\subsection{If Statement}

\subsection{While statement}
\verb|while expr: body| is translated as:
\begin{verbatim}
start:
(code for expr)
on_false end
(code for body)
jump start
end:
\end{verbatim}

\subsection{For statement}
\verb|for expr in seq: body|  is translated as:
\begin{verbatim}
    (code for seq)
    iter
    store_frame _temp
    protect end:
start:
    load_frame _temp
    next
    store_frame expr
    (code for body)
    jump start
end:
    end_protect
\end{verbatim}

\subsection{Try statement}
\begin{verbatim}
    try:
        body
    except:
\end{verbatim}


\subsection{With statement}
The `with' statement can be defined in terms of \verb|try finally|. See Python documentation for more details. 

\subsection{Function Declarations}
Function declarations consist of two parts: the function header and the code body.
The code body is translated as a sequence of statements plus a terminating \verb|return| in a separate code object, which is made available to the currently compiled code object as a constant.
Default values are explicitly converted into a tuple, annotations are explicitly converted into a dictionary.
The function header is translated as follows:
\begin{verbatim}
(code to create annotations dictionary)
(code to create tuple of default values)
make_function code_index
\end{verbatim}
Where \verb|code_index| is the index of the code object in the constants array.

\subsection{Class Declarations}
Class declarations consist of two parts: the class header and the code body. The class header declares the classes name, and its super-classes.
Class declarations are translated as follows:
\begin{verbatim}
(code to create super-class tuple)
new_scope
(code body)
pop_scope
make_class name_index
\end{verbatim}
Where \verb|name_index| is the index of the class name in the name array.

\end{document}
