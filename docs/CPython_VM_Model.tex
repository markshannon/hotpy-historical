\documentclass[a4paper,10pt]{article}

\newenvironment{myitemize}{
\begin{itemize}
  \setlength{\itemsep}{0pt}
  \setlength{\parskip}{0pt}
  \setlength{\parsep}{0pt}
}{\end{itemize}}


%opening
\title{}
\author{}

\begin{document}

\maketitle

\begin{abstract}

In order to optimize the performance of the CPython virtual machine, the semantics of each bytecode
must be well defined. In order to define these semantics, a model of execution is required. This paper describes such a model.
This model is an idealized version of the CPython virtual machine. The model can support the Python language as implemented by CPython,
but does so in subtly different ways. Before optimizing CPython, its implementation would need to be brought in line with the model.

\end{abstract}

\section{Overview}

The VM model consists of a single, global garbage-collected heap of objects and one or more machine-level threads of execution and one or more
Python threads. Each machine-level thread executes one Python thread at a time.
There is no fixed relation between machine-level threads and Python threads.
In the VM model, each machine-level thread consists of a triple of values, a \verb|thread_state| object, a data-stack and an  instruction-pointer.
When threads are suspended, the data-stack and instruction-pointer are stored in the \verb|thread_state| object, which encapsulates the entire state of the thread.

Each thread-state object corresponds one-to-one with a Python thread. This means that thread-switching can occur either when the underlying operating system switches the machine-level thread or when the VM changes the \verb|thread_state| object referred to by the machine-level thread.

Machine-level threads do not map one-to-one to Python threads. Allocation of objects and garbage-collection is automatically managed; memory management details are not part of the model.

\section{Objects}

Although Python can support an infinite number of classes,
the following classes are required to describe the execution model.

\begin{myitemize}
\item object
\item type
\item thread
\item thread\_state
\item frame
\item function
\item code\_object
\item exception\_handler
\end{myitemize}

The following classes are required  to describe the optimisation passes.
\begin{myitemize}
\item dict
\item dict\_values
\item dict\_keys
\item int
\item float
\end{myitemize}

\subsection{The object class}
The \verb|object| class is the root of the inheritance hierarchy in Python. All objects include the single \verb|class| field,
defined in the \verb|object| class. Many objects also have a \verb|dict| field which refers to the instance dictionary for that object. Whether or not an object has a \verb|dict| field is recorded in its \verb|class|.

\subsection{The type class}
Since Python is a dynamic language, classes are objects like all other objects. All classes have \verb|type| as their \verb|class|. \verb|type| is its own class. The \verb|type| class have various fields that are required for describing objects to the garbage collector and other book-keeping details. The following fields are relevant to the VM model.
\begin{description}
\item[dict\_offset] Records the offset of the \verb|dict| field in instances of this class, or zero if instances do not have dictionaries.
\item[dict] The \verb|dict| field of a \verb|type| object is special.
It describes the attributes of all instances of the class (\verb|type| object), rather than the class itself.
\item[mro] Abbreviation for method resolution order. An array of \verb|type| objects, used when resolving attributes. To \verb|obj.x| look for 'x' in \verb|obj.class.mro[0].dict|, then\verb| obj.class.mro[1].dict|, through to \verb|obj.class.mro[n-1].dict|. Where \verb|t.mro[0]| is \verb|t| and \verb|t.mro[n-1]| is the \verb|object| class.
\end{description}

\subsection{The thread class}

Each thread object contains information about the underlying machine-thread as well as thread-local variables.

The profiler and tracer are a required part of the model, as they strongly influence optimisations. Most of the state of a HotPy thread is stored in the stack of \verb|frame|s, implemented as a singly linked list. Each \verb|thread| object has a \verb|current_frame| field which refers to the currently executing frame. 

\subsection{The thread\_state class}

Each thread object consists of a large number of fields; many of these are necessary for Python semantics or are used by the optimisers. The fields relevent to the VM model are:
\begin{description}
\item[current\_frame] The frame for the currently executing function.
\item[stack\_top] Where top of data\_stack pointer is stored when suspended.
\item[next\_ip] Next instruction to be executed (only valid when suspended).
\item[exception] The exception currently being handled (if any).
\item[c\_exception] When an exception is generated in C code (rather than the interpreter), this value is set.
\item[call\_depth] The depth of the frame stack, used to prevent run away recursion.
\item[profiler] A callable object, usually used for proiling. It is called on function calls and returns, for each line of code executed, whenever an exception is raised.
\item[tracer] A callable object, usually used for debugging.
\end{description}

The profiler and tracer are a required part of the model, as they strongly influence optimisations. Most of the state of a Python thread is stored in the stack of \verb|frame|s, implemented as a singly linked list. Each \verb|thread_state| object has a \verb|current_frame| field which refers to the currently executing frame.

\subsection{The frame class}

The following fields of the \verb|frame| class are relevant to the execution model.
\begin{description}
\item[length] The number of local variables.
\item[locals\_array] The local variable array
\item[next\_ip] A pointer to the next instruction (bytecode) to be executed.
\item[line\_no] The current line number
\item[stack\_base] The top of stack pointer on entry to this frame. 
\item[exception\_handlers]  A singly linked list of exception handlers (may be empty).
\item[scopes] Holds references to built-in, module and local scopes.
\item[code] The code object being executed by this frame.
It also contains a \verb|dict| for the local variables, for those frames that cannot store their locals in the locals\_array.
\item[back] A reference to the caller frame.
\end{description}

\subsection{The function class}

The \verb|function| class describes a Python function. Most of the details of the  Python function are bound up in the \verb|code_object| object which includes the actual bytecodes, as well as the number of parameters and number of local variables. The \verb|function| contains those details that cannot be determined statically from the source code, such as default values.
The \verb|function| class has the following fields that are relevant to the HotPy VM model.
\begin{description}
\item[code] The code object for this function.
\item[defaults] Default values for unspecified parameters (a tuple).
\item[kw\_defaults] Default values for unspecified keyword-only parameters (a dict).
\item[scopes] The built-in and module-level dictionaries for this function.
\item[closure] If this is a nested function, this refers to the frame in which this function object was created.
\end{description}

\subsection{The code\_object class}

The \verb|code_object| class describes the static parts of a Python function. The  \verb|code_object| class has the following fields:
\begin{description}
\item[kind] A code describing the parameter passing conventions for this function. There are four possibilities: surplus positional arguments may raise an exception or be passed in a tuple; surplus named arguments may raise an exception or be passed in a dictionary.
\item[param\_count] The number of parameters declared for this function.
\item[locals\_count] the number of local variables declared in this function
\item[bytecodes] The bytecode instructions for this code\_object.
\end{description}

\subsection{The exception\_handler class}
The \verb|exception_handler| class is not a class in the standard Python library, it is purely internal to the VM.

The \verb|exception_handler| class has the following fields:
\begin{description}
\item[data\_stack\_depth] The number of values on the data stack
\item[ip] The first bytecode to be executed when this handler is invoked.
\item[next] The next \verb|exception_handler| object in the linked list.
\end{description}

\subsection{The scopes class}
The \verb|scopes|  class holds the local, module-level and global scopes for functions and frames.
It has the following fields:
\begin{description}
\item[builtins] The builtins dictionary
\item[globals] The globals (module-level) dictionary
\item[locals] The locals dictionary. This is NULL for scopes stored in functions and those frames which use ``fast'' locals.
\end{description}

\section{Raising Exceptions}

When an exception is raised (either in C code or in Python code), its traceback is set and if \verb|thread_state->exception| is non-NULL, then the \verb|__context__| of the exception is set to \verb|thread_state->exception|. In C code the \verb|thread_state->c_exception| is set to the exception, and an error code is returned until the interpreter is reached, at which point the \verb|thread_state->c_exception| is pushed to the stack. In Python code, the exception is already on the stack.
The 


\end{document}
